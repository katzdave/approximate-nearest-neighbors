\chapter{Conclusion} % Main chapter title

\label{conclusion} % For referencing the chapter elsewhere, use \ref{Chapter1} 

\lhead{Chapter 7. \emph{Conclusion}} % This is for the header on each page - perhaps a 

From our results in Chapter \ref{results}, we have shown that from a theoretical perspective our system has the potential for large performance gains over a conventional k-d tree for an ANN system which allows users to specify dimension relevance weights with each query.  For queries whose DRVs are drastically different than uniform, the gain of our system is substantial.  Our system also tended to scale better than a standard k-d tree as the dimension and size of datasets increased, making the performance gain more significant for larger datasets.

In Chapter \ref{futurework}, we discussed performance concerns for implementing a live system.  While our system does require linear memory consumption per tree, the actual memory consumption of each node can be kept very low if bit level indexing is used.  Additionally, our system has the potential to be distributed, allowing it to handle both large datasets and a large number of queries.  The key downside of our system is that in order to allow users to specify weights on each dimension, they must be tied to real world features, limiting the types of dimensionality reduction techniques which can be performed.

A variety of different possibilities exist for future work on the project.  The most important is to deploy it in a live system.  The data gathered from this system could be applied to developing a self tuning heuristic based on usage statistics.  Additionally, the information about the types of queries being performed could be used to develop more sophisticated heuristics for generating seed DRVs of trees.

Another key aspect to test is performance when compared to other types of indexes.  Because our testbed was not optimized for real time, only k-d tree based indexes were tested, comparing the number of nodes searched.  This metric does not carry over to hash and graph based indexes.  In order to test against these types of systems, a real time optimized implementation must be developed.

One last important aspect is to verify the usefulness to direct applications of allowing a DRV to be specified on each query.  While intuitively this change could fit directly into a recommendation system or information retrieval system, it is important to obtain feedback as to how willing users of a system would be to use this type of feature, and  how much of a perceived impact it has on their results.  Other possible applications of this type of query should also be explored and tested.  Overall, while we believe this type of system has the potential for large performance benefits, further testing in the aforementioned areas is necessary.